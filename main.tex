%%%%%%%%%%%%%%%%%%%%%%%%%%%%%%%%%%%%%%%%%%%%%%%%%%%%%%%%%%%%%%%%%%%%%%%%%%%%%%%%
% School of Engineering Fribourg - LaTeX Template                              %
% Florent Kilchoer - 2020                                                      %
% Special Thanks to Yannis Huber for theme inspiration                         %
%                                                                              %
% THIS IS NOT AN OFFICIAL TEMPLATE                                             %
%%%%%%%%%%%%%%%%%%%%%%%%%%%%%%%%%%%%%%%%%%%%%%%%%%%%%%%%%%%%%%%%%%%%%%%%%%%%%%%%

\documentclass[
    fr,
    twoside]{customTemplate}

%% Glossary & Acronyms
\makeglossaries

\newglossaryentry{esp8266}
{
    name=ESP8266,
    description={Circuit intégré à microcontrôleur avec connexion Wi-Fi développé par le fabricant chinois Espressif}
}

\newglossaryentry{heroku}
{
    name=Heroku,
    description={Heroku est une plate-forme en tant que services (\acrshort{paas}) qui permet aux développeurs de créer, exécuter et exploiter des applications entièrement dans le \emph{cloud}}
}

%% Pas plutôt un bas de page ?
\newglossaryentry{topic}
{
    name=\emph{topic},
    description={Arborescence permettant d'accéder à une ressource, utilisé par le protocol MQTT}
}

\newglossaryentry{docker}
{
    name=Docker,
    description={Docker est une plateforme de conteneurisation permettant aux développeurs de déployer du code ainsi que ses dépendances associées de manière simple et sans se soucier du système d'exploitation cible}
}

\newglossaryentry{serverless}
{
    name={\emph{Serverless computing}},
    description={Le \emph{serverless computing} est un paradigme de \emph{cloud computing} dans lequel le fournisseur de serveur gère dynamiquement les ressources allouées au service client. Le prix dépend des ressources effectivement consommées et non des capacités d'un serveur acheté à l'avance \cite{wiki-serverless}}
}

\newglossaryentry{cloud}
{
    name=\emph{cloud},
    description={Le \emph{cloud} consiste en des serveurs informatiques distants communiquants par l'intermédiaire d'un réseau, généralement Internet, pour stocker des données ou les exploiter \cite{wiki-cloud}}
}

\newglossaryentry{fog-computing}
{
    name={\emph{fog computing}},
    description={Le \emph{fog computing} consiste à exploiter des applications et des infrastructures de traitement et de stockage de proximité, servant d'intermédiaire entre des objets connectés et une architecture de \emph{cloud computing} classique \cite{wiki-fog-computing}}
}

\newglossaryentry{thing}
{
    name={\emph{thing}},
    description={Un \emph{thing} est une représentation d'un dispositif ou d'une entité logique spécifique. Il peut s'agir d'un dispositif ou d'un capteur physique. Dans notre projet, un \emph{thing} représente un \uc{}.}
}

\newglossaryentry{cicd}
{
    name={CI/CD},
    description={De l'anglais \emph{Continuous Integration/Continuous Delivery or Deployement} est un concept dans le développement logiciel qui vise à automatiser les tests et le déploiement du produit afin de minimiser le temps entre les nouvelles versions et de prévenir les erreurs dans le code. Souvent le CI/CD consiste à tester le code automatiquement lors de chaque commit et de le déployer lors d'un \emph{push} sur une certaine branche.}
}
%% ACRONYMS
\newacronym{iot}{IoT}{Internet des Objets}
\newacronym{gke}{GKE}{Google Kubernetes Engine}
\newacronym{http}{HTTP}{Hypertext Transfer Protocol}
\newacronym{mqtt}{MQTT}{Message Queuing Telemetry Transport}
\newacronym{lipo}{LiPo}{Lithium Polymer}
\newacronym{ap}{AP}{Access Point}
\newacronym{dns}{DNS}{Domain Name System}
\newacronym{dhcp}{DHCP}{Dynamic Host Configuration Protocol}
\newacronym{ftp}{FTP}{File Transfert Protocol}
\newacronym{paas}{PaaS}{Platform as a service}
\newacronym{aws}{AWS}{Amazon Web Services}
\newacronym{oidc}{OIDC}{OpenID Connect}
\newacronym{jwt}{JWT}{JSON Web Token}
\newacronym{lwt}{LWT}{Last Will and Testament}
\newacronym{orm}{ORM}{Object-relational mapping}
\newacronym{rest}{REST}{Representational state transfer}

%% REFERENCES
\addbibresource{sources.bib}

%% REVISIONS
\AddRevision{0.1.0}{23 février 2020}{Rédaction du cahier des charges}
%\AddRevision{1.0.0}{30 janvier 2020}{Rendu final}

%%%%%%%%%%%%%%%%%%%%%%%%%%%%%%%%%%%%%%%%%%%%%%%%%%%%%%%%%%%%%%%%%%%%%%%%%%%%%%%%

% Comment these lines to hide informations
%% DOCUMENT INFORMATION
\title{Title}
\subtitle{Subtitle}
%\course{Course}
\students{Florent Kilchoer}{}
\supervisors{Serge Ayer}{Jacques Supcik}


% Uncomment this line to use custom/no date (this replace the last revision date)
%\lastReleaseDate{\today}
%\lastReleaseDate{\empty}

\usepackage{xifthen}% provides \isempty test

\begin{document}


%%%%%%%%%%%%%%%%%%%%%%%%%%%%%%%%%%% DOCUMENT %%%%%%%%%%%%%%%%%%%%%%%%%%%%%%%%%%% 
\maketitlepage{}
% OPTIONAL : Display the revision table
\makerevisiontable{}
\cleardoublepage

% OPTIONAL : The abstract has to be written in abstract.tex
%\pagenumbering{Roman} 
\begin{abstract}
test.
\end{abstract}

% OPTIONAL : Uncomment the following line to add a table of content
\maketableofcontent{}
\cleardoublepage




%\smallheader
\fullheader
%\noheader

\pagenumbering{arabic} 

%\begin{comment}
\section{Code without Title}
\begin{code}{python}{}
String t
def max(int a, int b):
    pass
\end{code}
\begin{verbatim}
\begin{code}{python}{}
String t
def max(int a, int b):
    pass   
\end{code}
\end{verbatim}
\section{Code with Title}
\begin{code}{python}{Code with title}
String t
def max(int a, int b):
    pass
\end{code}
\begin{verbatim}
\begin{code}{python}{Code with title}
String t
def max(int a, int b):
    pass   
\end{code}
\end{verbatim}

\section{Highlitext text}
\subsection{Normal text}
\verb+\hl{Highlighted Text}+ : \hl{Highlighted Text}
\subsection{In code}
\begin{code}{python}{}
!\colorbox{green}{String t}!
def max(int a, int b):
    !\colorbox{yellow}{pass}!
\end{code}
\begin{verbatim}
\begin{code}{python}{}
!\colorbox{green}{String t}!
def max(int a, int b):
    !\colorbox{yellow}{pass}!
\end{code}    
\end{verbatim}



\section{Todo}

\verb+\todo[inline]{Text}+\\
\verb+\todo{Text}+\\
\verb+\todo[color=green]{Text}+

"Lorem ipsum dolor sit amet, consectetur adipiscing elit, sed do eiusmod tempor incididunt ut labore et dolore magna aliqua. Ut enim ad minim veniam, quis nostrud exercitation ullamco laboris nisi ut aliquip ex ea commodo consequat. \todo[inline]{An inline todo} Duis aute irure dolor in reprehenderit in voluptate velit esse cillum dolore eu fugiat nulla pariatur. Excepteur sint occaecat cupidatat non proident, sunt in culpa qui officia deserunt mollit anim id est laborum."
\todo{A todo note}
"Lorem ipsum dolor sit amet, consectetur adipiscing elit, sed do eiusmod tempor incididunt ut labore et dolore magna aliqua. Ut enim ad minim veniam, quis nostrud exercitation ullamco laboris nisi ut aliquip ex ea commodo consequat. Duis aute irure dolor in reprehenderit in voluptate velit esse cillum dolore eu fugiat nulla pariatur. Excepteur sint occaecat cupidatat non proident, sunt in culpa qui officia deserunt mollit anim id est laborum."
\todo[color=green]{And a green note}

\missingfigure{Add a picture here}

Missing figures : \verb+\missingfigure{Text}+

\listoftodos
List of todos : \verb+\listoftodos+


\section{Acronmys / Glossary}
Only used term are displayed in Acronyms/Glossary list

Example :
\begin{itemize}
    \item \verb+\acrshort{mqtt}+ : \acrshort{mqtt}
    \item \verb+\acrlong{mqtt}+ : \acrlong{mqtt}
    \item \verb+\acrfull{mqtt}+ : \acrfull{mqtt}
    \item \verb+\gls{cicd}+ : \gls{cicd}
\end{itemize}

\section{Source}
Add entry in sources.bib and use \verb+\cite{sourceName}+ to cite it \cite{sourceName}.
Use \verb+\printbibliography{}+ to print the bibliography :
\printbibliography{}

\section{Graphviz}
\graphviz{s01exa}{'
    graph G { 
        node[shape=rectangle]
        n0 [label="a(Z)"];
        n1 [label="prime(Z),a(8)"];
        n11 [label="a(8)"];
        n12 [label="a(8)"];
        n13 [label="a(8)"];
        s0 [label="SUCCES"];
        s1 [label="SUCCES"];
        s2 [label="SUCCES"];
        s3 [label="SUCCES"];
        n0 -- n1 [label="(X:Z)"];
        n0 -- s0 [label="(X:8)"];
        n1 -- n11[label="(Z:2)"];
        n1 -- n12[label="(Z:3)"];
        n1 -- n13[label="(Z:5)"];
        n11 -- s1;
        n12 -- s2;
        n13 -- s3;
    }'}{width=0.5\textwidth}{Custom label}
%\end{comment}

%%%%%%%%%%%%%%%%%%%%%%%%%%%%%%%%%%% DOCUMENT %%%%%%%%%%%%%%%%%%%%%%%%%%%%%%%%%%% 
%% Acronmys / Glossary / Sources
\newpage
\printglossary[type=\acronymtype]{}
\printglossary[]{}
\printbibliography{}
\end{document}